\documentclass[a4paper, twoside]{report}

%-------------------------------------

% Document type
\usepackage{tgsreqspec}
%\usepacakge(tgstechspec}
%\usepackage{tgstechnote}
%\usepackage{tgsproc}
%\usepackage{tgsguidelines}
%\usepackage{tgstestplan}

% Document information
\documentreference{RS-0007}
\projectcode{DMAC}
\documentmajorversion{1}
\documentminorversion{A}
\clientcode{TGS}
\documentauthor{\persons{DEL}{}}
\documentcontroller{\persons{RAC}{}}
\documentquality{\persons{DOC}{}}
\title{CAMP Property Editor}
\date{\today}

% Contractual document ?
%\contractualdocument
\notcontractualdocument

% Document status
\draftdocument
%\underapprovaldocument
%\publisheddocument
%\obsoletedocument

% Document classification
%\publicdocument
%\restricteddocument
\confidentialdocument

% Generate PDF information
\makepdfinfo

%-------------------------------------

\begin{document}

\maketitle

\chapter*{About this document}

\section*{Document history}

% To add a history entry use:
% \addhistory{version}{data}{author}{changes}
% Newer history entries first
\begin{historytable}
\end{historytable}

\section*{Related documents}

% To add a related document entry use:
% \adddocument{document}{version}{parts}{knowledge}
% document: the document reference, title, url...
% version: version of the document if any
% parts: parts of the document which is relevant (pages, sections...)
% knowledge: should be either "Required", or "Recommanded", or "Advised"
\begin{documenttable}
    \adddocument{CAMP RS-0001}{1.A}{}{Required}
\end{documenttable}

\tableofcontents

%-------------------------------------

\chapter{Introduction\label{sec:introduction}}

The CAMP Property Editor is a graphical component which allows to view and edit CAMP properties. The
component is designed to be embedded into an application which allows easy interaction with CAMP
objects.

The property editor is extendable to allow custom display and edit features according to CAMP property
types.

This document specifies the requirements for the CAMP Property Editor.

\chapter{Description of needs\label{sec:needs}}

Here is an unordered list of needs for the CAMP Property Editor component.

\begin{itemize}
    \item Based on Qt
    \item Allow synchronous viewing and editing of properties
    \item Allow cancellation of an editing operations
    \item Handle clean and dirty states:
    \begin{itemize}
        \item Clean state correspond typically to the state of the object the last time it was saved
        \item Dirty state correspond to an object which has been modified since the last time it was
saved; the modified properties must be emphasized
    \end{itemize}
    \item Handle all CAMP property types
    \item Allow customization/extension of the property viewer and editor
    \item Allow to group properties according to the class hierarchy
    \item Allow property filtering by property name and by tag name using regular expressions
    \item Ease a future implementation of editing the common properties of several objects inheriting a same base class
    \item Ease a future implementation of an asynchronous viewing and editing mode
\end{itemize}

\chapter{Overview\label{sec:overview}}

The CAMP Property Editor shows the CAMP properties of a CAMP object in a manner similar to the Qt
designer property editor which does it for Qt properties (fig. \ref{fig:qt-property-editor}).

\image[0.5]{fig:qt-property-editor}{Qt Designer Property Editor}{images/qt_property_editor}

The editor widget must be composed of two main parts:
\begin{itemize}
    \item A tree like structure displays the property names in a first column
    \item A second column displays the associated values and allows to edit them
\end{itemize}

The property names and values displaying and editing features can be fully customized as explained
later. The section \ref{sec:property_types} details the different viewers and editors for all supported
CAMP property types.

\section{Filtering and grouping}

The property editor must display all visible properties of a CAMP object. Moreover, it must be possible
to filter the properties according to their names or their tag names using regular expressions.

The property editor must allow to group the properties of a CAMP object according to the class
hierarchy of the object. The figures \ref{fig:without-classes} and \ref{fig:with-classes} show an
example of a CAMP object being displayed by the CAMP property editor with and without grouping. In
the first case, there is no classes information. Each sub-tree directly lists all the properties of the
corresponding CAMP object. In the second case, all the properties are placed under the meta-class
name in which they have been declared. This allows to group properties from the most general sub-set to the
most specific one.

\image[0.5]{fig:without-classes}{Simple property tree}{images/without_classes}
\image[0.5]{fig:with-classes}{Property tree with classes hierarchy}{images/with_classes}

\section{Displaying and editing properties}

According to the property type, the property name area and the property value must be fully
customizable (i.e. property name and value displaying and editing).

When a property is beeing edited, it must be possible to cancel the operation at any time. The
previous value is then restored.

The CAMP property editor must allow to identify what properties have been changed from the previous
\emph{clean state}. The clean state corresponds usually to the last time the object has been saved.
All the properties which differ from that state must be emphasized. This way, the user can easily
see what has been modified.

The displaying and editing of several CAMP objects at the same time will be present in a future
version. In this case, the CAMP property editor must list all the properties of their first common
meta-class ancestor. Thus, it the objects do not inherit a same base class, no property will be
listed. This feature is not required for now, but it must be taken into account.

\chapter{Handling property types\label{sec:property_types}}

This section will define how the different CAMP property types will be handled by default. This
includes how is represented the property name, the property value and how the property value is
edited. First, we will begin by the default handling if the property type is not handled yet,
or if it does not provide it own display. Then, for each property type, any special handling is
listed.

\section{Default}

This section define how properties will be displayed by default (i.e. when no overriding occurs).

The property name area display the property name as a simple text. If the property is dirty, the
text is bolded.

The property value area display the property value as a string if it is convertible into this
format. Otherwise, the text \emph{not available} is displayed in bold and italic. If the property is
not readable, the area is left empty.

If the property is not writable, the content of the area must be grayed.

There is no default editor neither for the property name area, nor for the property value one.

\section{Boolean}

The property value is displayed and edited with a check box.

\section{Integer}

The value editor is a spin box for integer number.

\section{Real}

The value editor is a spin box for real number.

\section{String}

The editor is a line edit.

\section{Array}

An array can be displayed in two ways:
\begin{description}
    \item[Step by step] In this mode, the property name area contains a spinbox which allows to
iterate over each value contained into the array. The corresponding value is displayed in the
property value area according to the element type, as usual. However, if the element are arrays too,
the value area will be left blank, and another spinbox will be displayed as a child node of the
property name one. If the element are objects, their properties are listed as child nodes too.
    \item[Expanded] In this mode, the whole content of the array is listed in a sub-tree. The
property value area shows the index of the value displayed in the value area.
\end{description}

The default behavior is the first one because it is more compact and it avoid to query all the array
values.

If the array is dynamic, a \emph{plus} and \emph{minus} buttons are displayed next to the spinbox to
allow adding and removing of element. The plus button will open a popup menu to select if the
insertion must occur before or after the current item (this will insert a default constructed
value). The minus button will remove the current item.

Right clicking the property name are will allow to switch from the step by step mode to the
expanded one for this property.

\section{Enumeration}

The editor for enumeration is a combo-box which displays all the possible value.

\section{Object}

The property value area displays the text representation of the object if it exists.
There is no editor. Instead, the properties of the object are listed in a sub-tree as usual.

\end{document}

