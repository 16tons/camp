
\documentclass[a4paper, twoside]{report}

%-------------------------------------

% Document type
\usepackage{tgstechspec}

% Document informations
\documentreference{TS-0001}
\projectcode{DMAC}
\documentmajorversion{1}
\documentminorversion{A}
\clientcode{TGS}
\notcontractualdocument
\documentauthor{\persons{GOM}{}}
\documentcontroller{\persons{DEL}{}}
\documentquality{\persons{DOC}{}}
\title{CAMP}
\date{February 19, 2009}

% Document status
%\draftdocument
%\underapprovaldocument
\publisheddocument
%\obsoletedocument

% Document classification
%\publicdocument
%\restricteddocument
\confidentialdocument

% Generate PDF information
\makepdfinfo

%-------------------------------------

\begin{document}

\maketitle

\chapter*{About this document}

\section*{Document history}

\begin{historytable}
    \addhistory{1.A}{2009-02-19}{GOM}{Initial version}
\end{historytable}

\section*{Related documents}

\begin{documenttable}
    \adddocument{RS-0001}{1.A}{}{Required}
\end{documenttable}

\tableofcontents

\chapter{Overview of design and implementation}

\section{Metaprogramming}

	Due to its high-level of genericity, CAMP makes an intensive use of C++ meta-programming technics.
	We choose to use a lot of compile-time code generation, automatic type detection, etc. to
	make the public API as intuitive and straight-forward as possible for the end user, who
	may not have particular meta-programming skills.
	
	By using meta-programming technics, we also ensure that metaclasses declarations are
	type-safe, and that CAMP will never be able to create valid objects that would be dangerous to use.
	
	To help writing meta-programs, the boost\footnote{\url{http://www.boost.org/}} library is used,
	especially its \verb@function@, \verb@bind@, \verb@function_types@, \verb@type_traits@
	and \verb@variant@ modules.
	
\section{Abstraction}

	Another aspect of CAMP is its flexibility at runtime, which means that all the templates
	must remain hidden from the public interface and be used only internally. As a consequence,
	CAMP is made of two layers: the abstract/typeless public interface, and the specialized/strongly-typed
	implementation.
	
	This is generally implemented using the same pattern: an
	abstract base class handling abstract variables (metavalue, metaobject),
	and template derived classes which convert those variables to the corresponding typed values
	and forward them to the actual member of the target C++ class.
	
	This organization is illlustrated in
	chapter \ref{sec:property} with the typical examples of metaproperties.
	
\section{Namespaces}

	CAMP defines its classes and functions in three namespaces:
	
	\begin{tabular}{r|l}
		\verb@camp@         & public API \\
		\verb@camp_ext@     & public API which can be extended by users \\
		\verb@camp::detail@ & private API \\
	\end{tabular}

\section{Macros}

	All the macros defined by CAMP have the \verb@CAMP_@ prefix.

\chapter{Metaclass}

	Three classes are involved in management of metaclasses:
	
	\begin{itemize}
		\item \verb@Class@ represents the metaclass; it contains the interface to access its properties, functions, etc.
		\item \verb@ClassManager@ is a singleton class which creates, stores and destroys the metaclasses instances
		\item \verb@ClassBuilder@ is a proxy which fills a metaclass with bases, properties, functions and constructors
	\end{itemize}
	
	These three classes are detailed on figure \ref{fig:class}.
	
	Creating a metaclass can be decomposed into the following steps:
	
\begin{cppcode}
template <typename T>
ClassBuilder<T> Class::declare(string name)
{
  // 1- get the CAMP identifier of T
  string id = staticTypeId<T>();
  
  // 2- register it and get back the new metaclass created by the manager
  Class& newClass = ClassManager::instance().registerNew(name, id);
  
  // 3- return a builder constructed from the new metaclass
  return ClassBuilder<T>(newClass);
}
\end{cppcode}

	\rotatedimage{fig:class}{The \texttt{camp::Class} class}{images/class}

\chapter{Metaenum}

	The implementation of metaenums follows the same pattern as metaclasses:
	
	\begin{itemize}
		\item \verb@Enum@ represents the metaenum; it contains the interface to access its items
		\item \verb@EnumManager@ is a singleton class which creates, stores and destroys the metaenums instances
		\item \verb@EnumBuilder@ is a proxy which fills a metaenum with $<$name, value$>$ pairs
	\end{itemize}
	
	These three classes are detailed on figure \ref{fig:enum}.
	
	Similarly to creating a metaclass, creating a metaenum can be decomposed into the following steps:
	
\begin{cppcode}
template <typename T>
EnumBuilder Enum::declare(string name)
{
  // 1- get the CAMP identifier of T
  string id = staticTypeId<T>();
  
  // 2- register it and get back the new metaenum created by the manager
  Enum& newEnum = EnumManager::instance().registerNew(name, id);
  
  // 3- return a builder constructed from the new metaenum
  return EnumBuilder(newEnum);
}
\end{cppcode}

	\rotatedimage{fig:enum}{The \texttt{camp::Enum} class}{images/enum}

\chapter{Metaproperty}
\label{sec:property}

	Metaproperties are implemented using a hierarchy of classes with three layers, as shown in
	figure \ref{fig:property}.
	
	The first layer is the main part of the public API: \verb@camp::Property@. It defines pure virtuals to get and set
	the value of the property, plus all the accessors not related to the implementation of the property. It also
	defines the \verb@isReadable@ and \verb@isWritable@ virtual functions, to let the derived classes tell whether or not
	it has valid getter and setter, independently of the readable and writable state of the property. This allows to handle properly
	properties marked as writable but not bound to a valid setter (same for readable/getter).
	
	The second layer is a less visible part of the API: \verb@camp::Property@ is derived in 4 classes that
	fit the different types of properties, and add the extra features related to them:
	
	\begin{itemize}
		\item \verb@SimpleProperty@ adds nothing, it is just separated from the other properties by design
		\item \verb@ArrayProperty@ adds array semantics through virtual functions to access the elements of the array
		\item \verb@EnumProperty@ adds an access to the metaenum of the property
		\item \verb@UserProperty@ adds an access to the metaclass of the property
	\end{itemize}
	
	This layer is still abstract, but completes the set of functions needed for metaproperties.
	
	The third layer is purely an implementation layer, it consists of the typed implementations of the 4 property classes.
	These derived classes are strongly typed, and can thus store and manipulate the actual C++ property.

	\rotatedimage{fig:property}{Implementation of metaproperties}{images/property}

\section{Array metaproperties}

	Array metaproperties (\verb@camp::ArrayProperty@) are a special case since they have a whole new semantic. They add functions
	for manipulating the underlying array, whatever its actual type is (\verb@T[]@, \verb@std::vector@, ...).
	
	To do so, \verb@ArrayProperyImpl@ uses the template class \verb@camp_ext::ArrayMapper@. \verb@ArrayMapper<T>@ is a utility class
	which adapts the particular interface of an array type to a common interface usable by \verb@ArrayProperyImpl@. Moreover,
	this structure can be specialized by the user in order to make its own array class compatible with CAMP.
	
	Here is a sample specialization for \verb@std::vector@:
	
\begin{cppcode}
namespace camp_ext
{
  template <typename T>
  struct ArrayMapper<std::vector<T> >
  {
    typedef T ElementType;
    
    static bool dynamic()
    {
      return true;
    }
    
    static std::size_t size(const std::vector<T>& arr)
    {
      return arr.size();
    }
    
    static const T& get(const std::vector<T>& arr, std::size_t index)
    {
      return arr[index];
    }
    
    static void set(std::vector<T>& arr, std::size_t index, const T& value)
    {
      arr[index] = value;
    }
    
    static void insert(std::vector<T>& arr, std::size_t index, const T& value)
    {
      arr.insert(arr.begin() + index, value);
    }
    
    static void remove(std::vector<T>& arr, std::size_t index)
    {
      arr.erase(arr.begin() + index);
    }
}
\end{cppcode}

\chapter{Metafunction}
\label{sec:function}

	Metafunctions are implemented using several typed classes inheriting from \verb@camp::Function@, as shown in
	figure \ref{fig:function}.
	
	When calling \verb@Function::call@, some common checks are done on the object and the parameters, then the call is forwarded
	to the \verb@execute@ virtual function. This function is implemented in every \verb@FunctionImpl@ derived class, which
	is specialized for a different number of arguments so that it can extract them from the provided list and call the underlying function properly.

	Derived classes store the underlying C++ function to call in an instance of \verb@boost::function@, so that it can hold all the supported
	callable entities (member function, free function, ...) without the need for more specializations to handle this.
	
	To return a proper value even when the underlying function returns nothing (\verb@void@), \verb@FunctionImpl::execute@ uses a
	structure which dispatches the call at compile-time according to the return type: if it is void then it returns \verb@Value::nothing@,
	otherwise it returns a value constructed from the returned value.
	
	Example for the specialization handling two arguments:

\begin{cppcode}
template <typename R, typename C, typename A1, typename A2>
class FunctionImpl<R (C, A1, A2)>
{
public:

  Value execute(const UserObject& object, const Args& args) const
  {
    return CallHelper<R>::call(m_function, object.get<C>(),
                               args[0].to<A1>(), args[1].to<A2>());
  }

private:

  boost::function<R (C, A1, A2)> m_function;
};
\end{cppcode}

	\rotatedimage{fig:function}{Implementation of metafunctions}{images/function}

\chapter{Metaconstructor}

	Metaconstructors, just like metaproperties and metafunctions, are implemented using an abstract base class and
	typed specializations, as shown on figure \ref{fig:constructor}.

	The base class contains two virtual functions:
	\begin{itemize}
		\item \verb@matches(args)@ is called when the owner metaclass needs to match a constructor with a set of arguments; this is the only
		      way to find which constructor to call according to the provided arguments
		\item \verb@create(args)@ creates an instance of the class with the provided arguments
	\end{itemize}
	
	The derived classes, which are specialized on the type and number of arguments, can then implement these two virtual functions.

	\rotatedimage{fig:constructor}{Metaconstructors}{images/constructor}

\chapter{Metavalue}

	Metavalues in CAMP are implemented in the \verb@camp::Value@ class.
	In fact, this class is just a wrapper which hides the usage of more complex
	sub-modules to implement its features.
	
	Figure \ref{fig:value} describes the \verb@camp::Value@ interface and its components.
	
	\rotatedimage{fig:value}{The \texttt{camp::Value} class and its components}{images/value}

\section{Storage and conversions}

	In order to be able to store any type of variable, \verb@camp::Value@ internally uses a variant
	(\verb@boost::variant@). This variant can store any type among a finite set of predefined types,
	which means that CAMP must be able to map any type of variable to one of these types.
	
	Figure \ref{tab:value-types} shows this set of predefined types, and how other types map to them.

	\begin{figure}
		\label{tab:value-types}
		\begin{center}
		\begin{tabular}{|l|l|l|}
			\hline
			\textbf{C++ type} & \textbf{CAMP type} & \textbf{storage type} \\
			\hline
			N/A 										& noType      & \verb@NoType@ 		\\
			\hline
			\verb@bool@ 						& boolType    & \verb@bool@ 				\\
			\hline
			\verb@[unsigned] char@ 	& intType			& \verb@long@ 				\\
			\verb@[unsigned] short@ & ~						& ~ 			 						\\
			\verb@[unsigned] int@ 	& ~						& ~ 			 						\\
			\verb@[unsigned] long@ 	& ~						& ~ 			 						\\
			\hline
			\verb@float@ 						& realType    & \verb@double@ 			\\
			\verb@double@ 					& ~						& ~ 			 						\\
			\hline
			\verb@const char*@ 			& stringType  & \verb@std::string@	\\
			\verb@std::string@ 			& ~						& ~ 			 						\\
			\hline
			all \verb@enum@ types 	& enumType 		& \verb@EnumObject@ 	\\
			\hline
			all other types 				& userType    & \verb@UserObject@ 	\\
			\hline
		\end{tabular}
		\caption{C++ types mapped to CAMP types}
		\end{center}
	\end{figure}

	This type mapping is implemented using the \verb@camp_ext::ValueMapper@ template class.
	\verb@ValueMapper<T>@ provides:
	\begin{itemize}
		\item a compile-time constant which identifies the CAMP type that T maps to (i.e. how it should be stored in the variant)
		\item a function to convert from T to the mapped type
		\item functions to convert from supported storage types to T (see figure \ref{tab:value-types})
	\end{itemize}

	CAMP implements specializations of \verb@ValueMapper@ in order to support all the primitive
	and standard types.
	
	Users can extend the mapping system, by creating specializations of the
	\verb@ValueMapper@ template for their own types, and defining the proper type mapping
	and conversion functions.

\section{Visitors}

	Metavalues support static visitation, which allows to get a direct access to the value stored
	in the metavalue.
	
  Static visitors are often better than the usual approach using a runtime dispatch with a single
  switch/case statement. First, the functions handling each type are clearly separated, and dispatch
  is done at compile-time. Another benefit is that no conversion is needed, the argument is directly
  the variable which is stored in the metavalue.
  
  This is even more important when using binary visitors, which involve double dispatch and are almost impossible to
  handle properly with switch/case statements.
	
	To implement value visitors, CAMP uses the visitation feature of
	\verb@boost::variant@, which already does all the dispatching job.
	
	Users can write their own visitors by inheriting from \verb@camp::ValueVisitor@, which
	itself inherits from \verb@boost::static_visitor@ in order to be compatible with \verb@boost::variant@.
	
	The visitation system is also used internally by CAMP, to implement most of the features of
	\verb@camp::Value@ : conversions, comparisons, ...

	Here is an exemple of a visitor which creates an editor according to the type of the value:

\begin{minipage}{\linewidth}
\begin{cppcode}
struct EditorFactory : public ValueVisitor<PropertyEditor*>
{
    PropertyEditor* operator()(bool value)
    {
        return new BooleanEditor(value);
    }

    PropertyEditor* operator()(long value)
    {
        return new IntegerEditor(value);
    }

    PropertyEditor* operator()(double value)
    {
        return new RealEditor(value);
    }

    PropertyEditor* operator()(const std::string& value)
    {
        return new StringEditor(value);
    }

    PropertyEditor* operator()(const camp::EnumObject& value)
    {
        return new EnumEditor(value);
    }

    PropertyEditor* operator()(const camp::UserObject& value)
    {
        return new UserEditor(value);
    }
};

camp::Value value;
PropertyEditor* editor = value.visit(EditorFactory());
\end{cppcode}
\end{minipage}

\chapter{Enum and user objects}

	Enum and user objects are wrappers for instances which have a metaenum or metaclass.
	
	Figure \ref{fig:enumuserobject} describes these two classes.

\section{Enum object}

  An enum object is composed of an enumerated value, stored as a long integer, and its owner metaenum. This allows to get extra information,
  such as the name associated to the value in the metaenum.
  
  Enum objects are used as the storage type for enumerated values in \verb@camp::Value@.

\section{User object}

	A user object is composed of a raw pointer to the source instance, stored as a typeless \verb@void*@ pointer,
	and its owner metaclass. This allows to get extra information, such as the name of the class of the object.
	
	With the owner metaclass, \verb@UserObject@ is also able to safely handle the conversion of the object back to its original type,
	or to one of its base types if any.
	
	User objects are an important feature of CAMP: they are used as argument type for every function that needs to manipulate
	an abstract object (\verb@Property::get@, \verb@Property::set@, \verb@Function::call@, ...).

	\image{fig:enumuserobject}{\texttt{EnumObject} and \texttt{UserObject} classes}{images/enumuserobject}

\chapter{Type identification}
\label{sec:identification}

	In order to map a C++ type to its corresponding metaclass or a metaenum, CAMP implements a strong
	type identification system. It supports both static type identification, and	dynamic type identification
	(for polymorphic objects).

	Figure \ref{fig:typeid} shows the structures involved in the type identification system.

	\image{fig:typeid}{Type identification}{images/typeid}

\section{Static identification}
	
	Every user-defined type (class or enum) declared with the \verb@CAMP_TYPE@
	macro is given a string identifier. This identifier is guaranteed to
	be unique and to always refer to the same type. In other words, there is a 1:1
	mapping between a C++ type and its CAMP identifier.
	
	What the \verb@CAMP_TYPE@ does, is to implement a specialization if the \verb@StaticTypeId@ template
	to provide a \verb@get@ function which returns the corresponding string identifier.
	This way, the type$\Leftrightarrow$id mapping is resolved during compilation, and any request for the metaclass
	or metaenum of a type that has not been declared with \verb@CAMP_TYPE@ can be detected and reported
	at compile-time (the generic version of \verb@StaticTypeId@ triggers an error in its \verb@get@ function).
	
	All this stuff is wrapped and accessed through the \verb@staticTypeId<T>()@ free function.

\section{Dynamic identification}

	In order to be able to retrieve the real dynamic type of a polymorphic object, two solutions exist:
	
	\begin{itemize}
		\item Use standard C++ RTTI through the \verb@typeid@ keyword
		\item Implement our own RTTI system through virtual functions
	\end{itemize}
	
	As standard RTTI is not realiable enough in this context, CAMP has to use its own RTTI system and thus
	inserts its own virtual function inside classes that want to support dynamic type identification.
	
	This is accomplished with the \verb@CAMP_RTTI@ macro, which expands to this code inside the target class:
	
\begin{cppcode}
virtual const char* campClassId() const
{
    return camp::detail::staticTypeId(this);
}
\end{cppcode}

	When requesting the type identifier of an object, CAMP detects through the
	\verb@HasDynamicTypeId@ structure if the C++ type of the object implements the \verb@CAMP_RTTI@ macro.
	If so, it can safely route the call to the \verb@campClassId@ function of the object; if not,
	it just retrieves its static type identifier.
	
	All this stuff is wrapped and accessed through the \verb@typeId(T)@ free function.

\chapter{Error handling}

	Errors in CAMP are implemented using exceptions. All types of errors inherit from a common base:
	\verb@Error@.
	
	\verb@Error@ inherit itself from the standard \verb@std::exception@ class. It also provides the
	execution context (source file and line number) where the error happened.
	
	Derived types of errors provide a generic error description, as well as information about the context
	of the error through accessors; see	figure \ref{fig:exceptions} for a detailed description of the error
	classes and their members.
	
	\image{fig:exceptions}{Error classes}{images/exceptions}

	Here is the list of CAMP error classes, and the functions that throw them:
	
	\begin{description}
		\item[InvalidClass] Class::declare, classByName, classByType, classByObject
		\item[InvalidEnum] Enum::declare, enumByName, enumByType, enumByObject
		\item[InvalidProperty] Class::property
		\item[InvalidFunction] Class::function
		\item[InvalidObject] UserObject::getClass, UserObject::get and every function taking a UserObject in argument
		\item[InvalidValue] Value::to, Value::operator T and every function taking a Value in argument
		\item[InvalidArgument] Function::call
		\item[InvalidEnumValue] Enum::name
		\item[InvalidEnumName] Enum::value
		\item[InvalidAccess] Function::call, Property::get, Property::set, ArrayProperty::get,
                         ArrayProperty::set, ArrayProperty::size, ArrayProperty::insert, ArrayProperty::remove
		\item[InvalidIndex] Every function taking an index in argument
	\end{description}

\chapter{Observer}

	\verb@Observer@ is a base class for classes which want to be notified of global CAMP events:
	
	\begin{itemize}
		\item Class creation
		\item Class removal
		\item Enum creation
		\item Enum removal
	\end{itemize}
	
	User-defined observers are registered and unregistered through the \verb@addObserver@ and \verb@removeObserver@ free functions,
	which forward to the notifier classes (\verb@ClassManager@ and \verb@EnumManager@). Observer notifiers inherit from the base
	class \verb@ObserverNotifier@, which implements all the common operations (registration, unregistration and iterating over the observers).
	
	Figure \ref{fig:observer} shows the implementation of the observer pattern in CAMP.

	\image{fig:observer}{Class and enum observer}{images/observer}

\chapter{Class visitor}

	The class visitor is made of a standard visitor design pattern, as shown on Figure \ref{fig:visitor}.
	
	A class which wants to explore the contents of a metaclass can inherit from the \verb@camp::ClassVisitor@ interface, and
	override the \verb@visit@ functions corresponding to the properties and functions types it wants to explore.
	
	Then, a call to \verb@visit(Class)@ starts the visitation and calls back the proper \verb@visit@ functions according to
	the contents of the visited metaclass.

	\image{fig:visitor}{Inner view of the class visitor framework}{images/visitor}

\newpage

\listoffigures

\end{document}
