
\documentclass[a4paper, twoside]{report}

%-------------------------------------

% Document type
\usepackage{tgsreqspec}

% Document informations
\documentreference{RS-0001}
\projectcode{DMAC}
\documentmajorversion{1}
\documentminorversion{A}
\clientcode{TGS}
\notcontractualdocument
\documentauthor{\persons{GOM}{}}
\documentcontroller{\persons{DEL}{}}
\documentquality{\persons{DOC}{}}
\title{CAMP}
\date{February 19, 2009}

% Document status
%\draftdocument
%\underapprovaldocument
\publisheddocument
%\obsoletedocument

% Document classification
%\publicdocument
%\restricteddocument
\confidentialdocument

% Generate PDF information
\makepdfinfo

%-------------------------------------

\begin{document}

\maketitle

\chapter*{About this document}

\section*{Document history}

\begin{historytable}
    \addhistory{1.A}{2009-02-19}{GOM}{Initial version}
\end{historytable}

%\section*{Related documents}

%\begin{documenttable}
%    \adddocument{RS-0001}{1.A}{}{Required}
%\end{documenttable}

\tableofcontents

\section{historytable environment}
\begin{historytable}
	\addhistory{0.1}{2009-02-19}{GOM}{first version}
\end{historytable}

\chapter{Abstract}

	CAMP is a C++ library whose aim is to provide reflection-like features for
	objects, i.e. to enable them to expose their attributes dynamically at runtime.
	This way, such objects can be manipulated in higher-level layers than direct 
	C++ code, and offer much more flexibility of usage and interaction.
	
	CAMP also provides a framework to help building components on top of class information
	like script engines, network transmitters, or XML I/O.
	
	This document specifies the requirements for the CAMP library.

\chapter{Description of needs}

	Here is an unordered list of needs for the CAMP library:

	\begin{itemize}
		\item CAMP is an independant, self-contained library, with no dependency on private projects
		\item CAMP must be as less intrusive as possible for classes exposing properties
		\item CAMP will provide an helper framework for writing higher-level components based on its features
		\item CAMP has no strong performance constraint, and should focus on flexibility and memory usage rather than speed
		\item An object using CAMP exposes two kinds of attributes:
		\begin{itemize}
			\item properties, that can be read and written
			\item functions, that can be executed
		\end{itemize}
		\item CAMP properties and functions must be bindable to any internal implementation in the host class
		\begin{itemize}
			\item a set of accessors
			\item a pointer to a member variable
			\item a member of a member
			\item other complicated binds with no limitation
		\end{itemize}
		\item CAMP properties consist of the following abstract types:
		\begin{itemize}
			\item primitive types: booleans, integers, reals
			\item strings
			\item enumerations
			\item arrays
			\item objects
		\end{itemize}
		\item The basic types of properties must be able to handle all the corresponding C++ primitive types
		\item A class bound to CAMP is automatically a valid type of property (type "object")
		\item A class not bound to CAMP cannot be a valid type of property, or function argument / return value
		\item CAMP properties are typed, but allow conversions from other types when they are compatible
		\item A CAMP property can be marked as readable and/or writable, and these states are specific to each object instance
		\item A CAMP function can be marked as callable, and this state is specific to each object instance
		\item Properties and functions must be able to hold custom tags, which can depend or not on objects instances
		\item The CAMP runtime information about an object must be directly accessible, given the object
		\item CAMP must be able to bind C++ enumerations, and provide specific features for them
		\item CAMP classes, enumerations, properties and functions must be accessed both by name and by index
		\item CAMP must provide a detailed error reporting system
		\item CAMP must support single and multiple inheritance
		\begin{itemize}
			\item an inherited object must inherit its parent's properties and functions
		\end{itemize}
	\end{itemize}

\chapter{Proposal}

\section{Overview}

	Figure \ref{fig:overview} is a high level overview of the CAMP system, showing its main classes and
	how they are related to each other.

	\image{fig:overview}{Overview of the CAMP library}{images/overview}

	\begin{description}
		\item[Metaclass (chapter \ref{sec:metaclass})]
			This is the entry point of the system; it basically	represents a C++ class in the CAMP system.
			A metaclass owns a set of metaproperties and a set of metafunctions.
		\item[Metaenum (chapter \ref{sec:metaenum})]
			Metaenums are the representation of C++ enumerated types in CAMP. They hold a set of
			$<$name, value$>$ pairs describing the corresponding C++ enum.
		\item[Metaproperty (chapter \ref{sec:metaproperty})]
			A metaproperty is the abstract representation of a member data in the C++ class,
			which can be read and written.
		\item[Metafunction (chapter \ref{sec:metafunction})]
			A metafunction is the abstract representation of a member function in the C++ class,
			which can be executed.
		\item[Visitor (chapter \ref{sec:visitor})]
			Even though properties and functions can be accessed easily through the base API, CAMP provides a slightly
			higher-level framework to facilitate manipulation of metaclasses.
	\end{description}

	\ibinfo
	{
			Metaproperties and metafunctions may be referred simply as ``properties'' and ``functions'' in the following sections
			of this document.
	}

\section{Metaclass}
\label{sec:metaclass}

	Metaclass is the main concept of the CAMP library. Declaring a metaclass for a
	C++ class allows it to expose its members to the CAMP system.
	
	Each metaclass instance describes a C++ class that has been exposed, and gives
	access to its properties and functions. It can then be used	to manipulate the
	class' information at runtime, and modify any instance of the class through
	the properties and functions.

\subsection{Declaration}

	A metaclass declaration contains the following elements in order to describe a class:
	\begin{itemize}
		\item Its target C++ class
		\item Its name
		\item The metaclasses that it inherits from
		\item Its constructors
		\item Its properties (see section \ref{sec:metaproperty})
		\item Its functions (see section \ref{sec:metafunction})
	\end{itemize}

	The target C++ class and the metaclass name are mandatory, all other elements are optional.
	
	A metaclass declaration must be completed in a single step, and cannot be modified later anymore.
	
	Example:

\begin{cppcode}
declare<TargetClass>("Name")
  .base<BaseClass>()
  .constructor<int>()
  .property("p", &TargetClass::p)
  .function("f", &TargetClass::f)
  ;
\end{cppcode}

	\ibwarning
	{
			Metaclasses are unique, and any attempt to declare a metaclass with the same name
			as an already existing metaclass must be rejected. \\
			However, declaring multiple metaclasses bound to the same C++ class is allowed.
	}
	
	\ibwarning
	{
		Property and function names have to be unique within the same metaclass. Declaring a property
		or function whose name already exists in the metaclass must be rejected.
	}

	Users must have the ability to bind public, protected and private members of the target class.
	Binding public members should be completely non-intrusive, while binding protected and private members
	may have to be a little bit intrusive regarding the target class.

\subsubsection{Mapping external systems}

	Some classes bound to CAMP, from other libraries, may already have a similar system
	providing meta-information about the class properties and functions. A perfect example
	would be Qt and its properties system.

	In such case, CAMP must be able to automatically bind this information to its own structure,
	without having to manually duplicate their definition for each class. This must be done at
	declaration-time.
	
	Example:

\begin{cppcode}
declare<TargetClass>("Name")
  .external<Mapper>()
  ;
\end{cppcode}
	
	Here, \verb@Mapper@ is a user-defined class which is responsible for
	translating custom meta-information to CAMP properties and functions.

\subsection{Retrieval}

	CAMP must provide different ways to retrieve a metaclass:

	\begin{tabular}{r|l}
		From a class type 		& \verb@classByType<MyClass>()@  \\
		From a metaclass name & \verb@classByName("MyClass")@  \\
		From a class instance &	\verb@classByObject(instance)@ \\
		From an index				  &	\verb@classByIndex(1)@ 				 \\
	\end{tabular}
	
	Each metaclass is represented by a unique instance; thus every call to one of the above functions
	to request the same metaclass will return a reference to the same instance.
	
	The returned object must be a \emph{read-only} reference to the metaclass; after declaration-time
	metaclasses cannot be modified, they can only be read.

	In the special case a C++ class is bound to multiple metaclasses, it must be possible to
	retrieve all of them using their total count and an indexed access:

	\begin{tabular}{r|l}
		From a type 				  & \verb@classByType<MyClass>(index)@    \\
		From a class instance & \verb@classByObject(instance, index)@
	\end{tabular}
	
	Retrieving a metaclass triggers an error if the metaclass does not exist. Such error should be reported
	at compile-time whenever possible.

\subsubsection{Object types}

	When getting a metaclass from an object, CAMP must accept any kind of storage/modifier on the actual type:

	\begin{itemize}
		\item A raw instance (\verb@T@)
		\item A const instance (\verb@const T@)
		\item A pointer (\verb@T*@)
		\item A const pointer (\verb@const T*@)
		\item A smart pointer (\verb@shared_ptr<T>@)
	\end{itemize}
	
	In all the above cases, CAMP must return the metaclass of \verb@T@.

\subsubsection{Polymorphic objects}

	When using a polymorphic object, CAMP must be able to use some kind of
	RTTI (\emph{RunTime Type Information}) to access the metaclass of the actual
	dynamic type of the object, not its static one.
	
\begin{cppcode}
Base* obj = new Derived;
metaclass = classByObject(obj);
\end{cppcode}
	
	In the above example, \verb@metaclass@ must be the metaclass of Derived.

\subsection{Usage}

	The interface of metaclasses must provide full access to the class members:
	
	\begin{itemize}
		\item Iterate on base metaclasses
		\item Iterate on properties
		\item Iterate on functions
		\item Get a property by its name
		\item Get a function by its name
		\item Check if the metaenum contains a property or a function by its name
		\item Construct and destroy instances of the target class
	\end{itemize}

\section{Metaenum}
\label{sec:metaenum}

	As C++ enumerated types are supported by CAMP, they must have an abstract representation in the system.
	This is the purpose of metaenums.
	
	A metaenum is basically a set of $<$name, value$>$ pairs that can be manipulated dynamically.
	The main feature of metaenums is that their items can be browsed at runtime and accessed by index,
	by name or by value.

\subsection{Declaration}

	A metaenum declaration is similar to a metaclass declaration, except that the only thing to bind are
	$<$name, value$>$ pairs.
	
\begin{cppcode}
declare<TargetEnum>("Name")
  .value("a", a)
  .value("b", b)
  .value("c", c)
  ;
\end{cppcode}
	
	\ibwarning
	{
			Metaenums are unique, and any attempt to declare a metaenum with the same name
			as an already existing metaenum must be rejected. \\
			However, declaring multiple metaenums bound to the same C++ enum is allowed.
	}
	
	\ibwarning
	{
		Names and values have to be unique within the same metaenum. Declaring a pair whose
		value or name already exists in the metaenum must be rejected.
	}

\subsubsection{Mapping external systems}

	Like for metaclasses, some enumerations bound to CAMP may already have a similar system
	providing meta-information about the enumeration content.

	In such case, CAMP must be able to interpret this meta-information and fill the metaenum
	from it.
	
\begin{cppcode}
declare<TargetEnum>("Name")
  .external<Mapper>();
\end{cppcode}
	
	Here, \verb@Mapper@ is a user-defined class which is responsible for
	translating custom meta-information to CAMP metaenum pairs.

\subsection{Retrieval}

	CAMP must provide different ways to retrieve a metaenum:

	\begin{tabular}{r|l}
		From an enum type			& \verb@enumByType<MyEnum>()@   \\
		From a metaenum name  & \verb@enumByName("MyEnum")@   \\
		From an enum instance &	\verb@enumByObject(instance)@ \\
		From an index				  &	\verb@enumByIndex(1)@ 			  \\
	\end{tabular}
	
	Each metaenum is represented by a unique instance; thus every call to one of the above functions
	to request the same metaenum will return a reference to the same instance.
	
	The returned object must be a \emph{read-only} reference to the metaenum; after declaration-time
	metaenums cannot be modified, they can only be read.

	In the special case a C++ enum is bound to multiple metaenums, it must be possible to
	retrieve all of them using their total count and an indexed access:

	\begin{tabular}{r|l}
		From a type 				  & \verb@enumByType<MyClass>(index)@    \\
		From an enum instance & \verb@enumByObject(instance, index)@
	\end{tabular}
	
	Retrieving a metaenum triggers an error if the metaenum does not exist. Such error will be reported
	at compile-time whenever possible.

\subsection{Usage}

	The interface of metaenums must provide full access to the enumeration members:
	
	\begin{itemize}
		\item The total number of pairs contained in the metaenum
		\item Iterate on pairs
		\item Get a value from its name
		\item Get a name from its value
		\item Check if the metaenum contains a name or a value
	\end{itemize}

\section{Metavalue}
\label{sec:metavalue}

	Metavalue is a high-level concept of variables. It is a kind of variant, that can hold values of
	any valid type (see section \ref{sec:metavalue-types}). It is used in CAMP to wrap C++ values
	into an abstract representation that will be manipulated by other CAMP entities (properties, functions, ...).
	
	Metavalues should behave as much as possible like C++ variables, but with extra flexibility.
	Here are the requirements for them:

	\begin{tabular}{r|l}
		\textbf{Requirement}									& \textbf{Example}												\\
		\hline
		Construct empty values 								& \verb@Value v;@    											\\
		Construct and assign from valid types	& \verb@Value v(5); v = true;@ 						\\
		Copy and assign from other values 		& \verb@Value v1; Value v2(v1); v1 = v2;@ \\
		Retrieve its abstract type 						& \verb@type = v.type();@ 							\\
		Check conversion to valid types 			& \verb@bool b = v.isConvertible<int>();@ \\
		Convert implicitly to valid types 		& \verb@int x = v;@ 											\\
		Convert explicitly to valid types 		& \verb@int x = v.to<int>();@ 						\\
		Support comparisons with other values	& \verb@ ==, !=, <, >, <=, >=@ 						\\
		Support type-safe visitation 					& \verb@see "visitation" section below@ 	\\
	\end{tabular}

\subsection{Supported types}
\label{sec:metavalue-types}

	Metavalues must support construction and conversion from/to the following set of types:

	\begin{description}
		\item[Booleans] bool
		\item[Integers] char, short, int, long and their signed/unsigned versions
		\item[Reals] float, double
		\item[Strings] const char*, std::string, and user-defined string types
		\item[Enumerations] user-defined enum types
		\item[User types] user-defined class types
	\end {description}
	
	By default CAMP provides mapping and conversion for primitive and standard types, but
	it must be extensible in order to support user-defined types. The extension system must
	allow users to provide the following information	about a custom type:
	
	\begin{itemize}
		\item The abstract CAMP type it maps to (integer, string, enumeration, ...)
		\item How to convert from it to the CAMP type
		\item How to convert from \emph{all} supported CAMP types to it
	\end{itemize}

\subsection{Type-safe visitation}
\label{sec:metavalue-visitor}

	Metavalues hide their actual C++ type in order to provide a uniform abstraction, but
	it may sometimes be desirable to have a direct access to the underlying type for operations
	that require strong type-safety.
	
	To enable this, metavalues have to support static visitation. A visitor
	must define a callback function for every possible type used by metavalues,
	and CAMP will call the proper one according to the actual type of the value.
	
	This can be seen as a compile-time switch, which dispatches the execution to
	the proper function according to the real type of the value.

	Example:
	
\begin{cppcode}
class Visitor : public ValueVisitor
{
public:

  void visit(Value::BoolType value);
  void visit(Value::IntegerType value);
  void visit(Value::StringType value);
  ...
};

Value v = 5;
visitor.visit(Visitor());
\end{cppcode}
	
	In this example, the metavalue \verb@v@ is assigned an integer so the visitor will be notified
	in its \verb@visit(Value::IntegerType)@ function, which directly receives the value as an integer type.
	
	\ibinfo
	{
		If visiting an empty value, the visitor must receive an argument of a special type (\texttt{Value::EmptyType})
		indicating that the source value contains nothing.
	}

	CAMP must support unary visitation as well as binary visitation.
	
	Here is an example of a binary visitor:

\begin{cppcode}
class BinaryVisitor : public ValueVisitor
{
public:

  void visit(Value::BoolType v1, Value::BoolType v2);
  void visit(Value::IntegerType v1, Value::StringType v2);
  ...
};

Value v = 5;
Value other = "hello";
v.visit(BinaryVisitor(), other);
\end{cppcode}
	
	In this example, the visitor will be notified	in its
	\texttt{visit(Value::IntegerType, Value::StringType)} function,
	which will directly receive 5 and "hello".

\section{Metaobject}
\label{sec:metaobject}

	Metaobject is a high-level concept of class instance. It is an abstract wrapper that can
	store any type of object, and which can be used by metaproperties and metafunctions
	with no knowledge about the actual C++ type of the stored object.
	
	Metaobjects should behave as much as possible like class instances, but with extra flexibility.
	Here are the requirements for them:

	\begin{tabular}{r|l}
		\textbf{Requirement}										& \textbf{Example}													\\
		\hline
		Construct and assign from any type	 		& \verb@Object o(obj); o = obj;@ 						\\
		Copy and assign from other objects 			& \verb@Object o1; Object o2(o1); o1 = o2;@ \\
		Retrieve its metaclass 									& \verb@metaclass = o.metaclass();@ 				\\
		Convert explicitly to a class type			& \verb@o.to<MyClass>();@ 									\\
		Support comparisons with other objects	& \verb@ ==, !=, <, >, <=, >=@ 							\\
	\end{tabular}

\subsection{Conversions from/to user types}
\label{sec:metaobject-conversion}

	Just like the metaclass retrieval function, the metaobject constructor accepts source objects
	with any kind of modifier/storage. Here is the list of what is accepted:

	\begin{itemize}
		\item A direct object (\verb@T@)
		\item A const object (\verb@const T@)
		\item A pointer to the object (\verb@T*@)
		\item A const pointer to the object (\verb@const T*@)
		\item A smart pointer to the object (\verb@shared_ptr<T>@)
	\end{itemize}
	
	In all the above cases, the metaobject will be aware that it is storing an object of type T, and its metaclass
	must be the one associated to T.

  Similarly, conversions from objects to user types using any of the above modifiers must be properly interpreted
  and return the expected result.
	
\subsection{Copy}
\label{sec:metaobject-copy}

	A metaobject must ensure that the stored object is never copied. In other words, metaobjects must always
	deal with direct pointers to the source objects, and never create new instances that the user may not be aware of.
	
	This has the consequence that users must ensure that the lifetime of a stored object is almost as long as
	the lifetime of the metaobject storing it.

\section{Metaproperty}
\label{sec:metaproperty}

	A metaproperty is a member of a metaclass, just like a variable can be a member of a C++ class. It abstracts
	the way its value is read and written, thus providing a unified way of accessing any property of an object.

	A metaproperty describes a property inside a metaclass, and is not bound to a particular object. A metaproperty
	must be used with an object instance of the owner metaclass to actually get or set its value.

	Metaproperties must provide the following interface:

	\begin{itemize}
		\item Retrieve its name
		\item Retrieve its type (see section \ref{sec:metaproperty-types})
		\item Check if the property is readable
		\item Check if the property is writable
		\item Get its value for a given object
		\item Set its value for a given object
		\item Implement tags (see section \ref{sec:tag})
	\end{itemize}

\subsection{Metaproperty types}
\label{sec:metaproperty-types}

	Every metaproperty has an abstract type, which refers to the actual type of the underlying
	C++ property.

	Here is the list of all the types that must be supported by metaproperties:

	\begin{itemize}
		\item Boolean
		\item Integer
		\item Real
		\item String
		\item Array
		\item Enumeration
		\item Object
	\end{itemize}

\subsection{Declaration}

	Metaproperties are defined within the declaration of their owner metaclass.
	
	A metaproperty declaration must provide the following information:
	
	\begin{itemize}
		\item Its name (must be unique within the metaproperties of a metaclass)
		\item The C++ variable / function it is bound to
		\item Its readable state (i.e. can we get the property?)
		\item Its writable state (i.e. can we set the property?)
		\item Its tags
	\end{itemize}
	
	The metaproperty name and its C++ binding are mandatory, all other attributes are optional. By default,
	a metaproperty should be readable if it is bound to a valid getter, writable if it is bound to a
	valid setter, and contain no tag.
	
	To access its underlying value, the metaproperty must be bindable to several C++ entities related to the target class
	(member functions, member variables, etc.). See section \ref{sec:metaproperty-bind} for a complete reference
	of all the possible bindings.

	The readable and writable states of a metaproperty can be bound either to a boolean constant,
	or to a function of the target C++ class returning a boolean. In such case, the state will depend
	on an object's state rather than being a constant value.
	
	\begin{minipage}{\linewidth}
		Example:

\begin{cppcode}
declare<TargetClass>("Name")
  .property("x", &TargetClass::x)
    .readable(true)
    .writable(false)
    .tag(scriptable)
  ...
\end{cppcode}
	
	\end{minipage}

	\ibinfo
	{
		Please refer to section \ref{sec:tag} for a complete description of tags.
	}

\subsection{Bindings}
\label{sec:metaproperty-bind}

	To get or set its value for a particular object, a metaproperty must be bindable to
	either a single getter, a single setter, or a pair of getter and setter.

\subsubsection{Getters}

	The getter must accept the following C++ entities:

	\begin{tabular}{r|l}
		\textbf{Binding}	& \textbf{Examples}											\\
		\hline
		A member variable	& \verb@int Class::x@ 			    				\\
		\hline
		A member function	& \verb@bool Class::get()@ 							\\
		~ 								& \verb@bool Class::get() const@ 				\\
		\hline
		A free function		& \verb@float get(Class&)@ 								\\
		~ 								& \verb@float get(Class*)@ 								\\
		~ 								& \verb@float get(const Class&)@ 					\\
		~ 								& \verb@float get(const Class*)@ 					\\
		\hline
		A function object	& \verb@boost::function<int (Class&)>@ 	\\
		~ 								& \verb@boost::bind(&Class::get, _1)@ 	\\
		\hline
	\end{tabular}

	Function objects are special objects behaving as functions by overloading an \verb@operator()@.
	The main purpose of supporting such objects is to make CAMP compatible with third-party
	libraries which provide powerful function semantics through function objects (such as
	std::binder, boost::function, boost::bind, etc.).

	If a metaproperty is bound to a getter which gives a write access to the
	variable (i.e. returning a non-const reference/pointer to the value), it must be
	considered as	a setter as well.

\subsubsection{Setters}

	The setter must accept the following C++ entities:

	\begin{tabular}{r|l}
		\textbf{Binding}	& \textbf{Examples}														\\
		\hline
		A member variable	& \verb@int Class::x@ 			    							\\
		\hline
		A member function	& \verb@void Class::set(bool)@ 								\\
		\hline
		A free function		& \verb@void set(Class&, float)@ 								\\
		~ 								& \verb@void set(Class*, float)@ 								\\
		\hline
		A function object	& \verb@boost::function<void (Class&, int)>@	\\
		~ 								& \verb@boost::bind(&Class::set, _1, _2)@ 		\\
		\hline
	\end{tabular}

\subsubsection{Returned value}

	The value returned by the binding can have the following modifiers:

	\begin{itemize}
		\item A direct value (\verb@T@)
		\item A const value (\verb@const T@)
		\item A pointer to the value (\verb@T*@)
		\item A const pointer to the value (\verb@const T*@)
		\item A smart pointer to the value (\verb@shared_ptr<T>@)
	\end{itemize}

	All of the above types must produce a metaproperty of type T, and must not
	alter the way the metaproperty is manipulated from the user's point of view.

\subsection{Special types of metaproperties}
\label{sec:metaproperty-derived}
	
	All types of properties share the same basic semantics (accessing a single value), but some of them have
	additional semantics due to their type. These types are Array, Enumeration and Object.
	
	In order to provide the extra features tied to their type, they must extend the basic metaproperty interface.

	Figure \ref{fig:properties} shows these special types of properties and the extra members they provide.

	\image{fig:properties}{The properties classes}{images/properties}

\subsubsection{Array property}
\label{sec:metaproperty-array}

	Array properties (i.e. properties bound to C++ arrays) extend the basic metaproperty interface
	to add array-related features. CAMP must handle both static arrays (their size is fixed)
	and dynamic ones (they can grow at runtime).
	
	An array metaproperty provides the following extra functions:

	\begin{itemize}
		\item Check whether the array is static or dynamic
		\item Retrieve its current size
		\item Get an element by its index
		\item Set an element by its index
		\item Insert a new element
		\item Remove an existing element by its index
	\end{itemize}
	
	Insertion and removal are available only if the array is dynamic, and must trigger an error if used on static arrays.

\par{\textbf{Extending array properties}}

	By default, CAMP must be able to handle the standard types of array-like containers:
	
	\begin{itemize}
		\item T[N]
		\item std::vector$<$T$>$
		\item std::list$<$T$>$
		\item boost::array$<$T, N$>$
	\end{itemize}
	
	But it may sometimes be desirable to be able to use a custom type of array, like a \verb@QVector<T>@ for instance.
	CAMP must provide	a simple way to extend the array metaproperties to any of these user-defined array types.

\subsubsection{Enum property}
\label{sec:metaproperty-enum}

	Enumerated properties (i.e. properties bound to C++ enum types) have the same semantics as simple properties,
	except they also provide information related to their corresponding metaenum (see section \ref{sec:metaenum}).

\subsubsection{Object property}
\label{sec:metaproperty-object}

	Object properties (i.e. properties bound to user-defined class types) have the same semantics as simple properties,
	except they also provide information related to their corresponding metaclass (see section \ref{sec:metaclass}).

\section{Metafunction}
\label{sec:metafunction}

	A metafunction is a member of a metaclass, just like a function can be a member of a C++ class. It abstracts
	the way it is executed, thus providing a unified way of calling any function of an object.

	A metafunction describes a function inside a metaclass, and is not bound to a particular object. A metafunction
	must be used with an object instance of the owner metaclass to actually execute a function.

	Metafunctions must provide the following interface:

	\begin{itemize}
		\item Retrieve its name
		\item Retrieve its return type (see section \ref{sec:metaproperty-types})
		\item Retrieve its arguments types (see section \ref{sec:metaproperty-types})
		\item Retrieve its arguments count
		\item Check if the function is callable
		\item Call the function for a given object
		\item Implement tags (see section \ref{sec:tag})
	\end{itemize}

\subsection{Declaration}

	Metafunctions are defined within the declaration of their owner metaclass.
	
	A metafunction declaration must provide the following information:
	
	\begin{itemize}
		\item Its name (must be unique within the metafunctions of a metaclass)
		\item The C++ function it is bound to
		\item Its callable state (i.e. can we execute the function?)
		\item Its tags
	\end{itemize}
	
	The metafunction name and its C++ binding are mandatory, all other attributes are optional. By default,
	a metafunction should be callable and contain no tag.
	
	To access its underlying function, the metafunction must be bindable to several C++ entities related to the target class
	(member functions, etc.). See section \ref{sec:metafunction-bind} for a complete reference
	of all the possible bindings.

	The callable state of a metafunction can be bound either to a boolean constant,
	or to a function of the target C++ class returning a boolean. In such case, the state will depend
	on an object's state rather than being a constant value.

	Example:

\begin{cppcode}
declare<TargetClass>("Name")
  .function("f", &TargetClass::f)
    .callable(true)
    .tag(scriptable)
  ...
\end{cppcode}
	
	\ibinfo
	{
		Please refer to section \ref{sec:tag} for a complete description of tags.
	}

\subsection{Bindings}
\label{sec:metafunction-bind}

	To execute its function for a particular object, a metafunction must be bindable to
	the following C++ types:

	\begin{tabular}{r|l}
		\textbf{Binding}	& \textbf{Examples}												\\
		\hline
		A member function	& \verb@int Class::f()@ 			    				\\
		~ 								& \verb@int Class::f() const@ 						\\
		\hline
		A free function		& \verb@bool f(Class&)@ 									\\
		~ 								& \verb@bool f(Class*)@ 									\\
		~ 								& \verb@bool f(const Class&)@ 						\\
		~ 								& \verb@bool f(const Class*)@ 						\\
		\hline
		A function object	& \verb@boost::function<float (Class&)>@ 	\\
		\hline
	\end{tabular}

	Function objects are special objects behaving as functions by overloading an \verb@operator()@.

\subsubsection{Result and arguments types}

	A metafunction can be bound to a function returning a value or nothing (\verb@void@).
	It must at least support functions with up to 5 arguments.

	The result and arguments types of the binding can have the following modifiers:

	\begin{itemize}
		\item A direct value (\verb@T@)
		\item A const value (\verb@const T@)
		\item A pointer to the value (\verb@T*@)
		\item A const pointer to the value (\verb@const T*@)
		\item A smart pointer to the value (\verb@shared_ptr<T>@)
	\end{itemize}

	All of the above types must be interpreted the same way as T, and must not
	alter the way the metafunction is manipulated from the user's point of view.

\section{Properties and functions tags}
\label{sec:tag}

	In addition to their fixed attributes (readable, writable, callable),
	metaproperties and metafunctions can be extended to hold custom tags. This allows to communicate with
	external high-level components without impacting the CAMP API. An example would be a "persistant"
	tag, to mark which properties have to be serialized by an XML writer or a network engine.
	
	A tag can be:

	\begin{tabular}{r|l}
		\hline
		A single static value								& \verb@"persistent"@ 													\\
		~																		& \verb@scriptable@ 			  										\\
		\hline
		A static $<$key, value$>$ pair			& \verb@<"category", 5>@ 												\\
		~																		& \verb@<designable, true>@ 										\\
		\hline
		A key associated to a dynamic value	& \verb@<"category", &TargetClass::category>@ 	\\
		~																		& \verb@<designable, &TargetClass::designable>@ \\
		\hline
	\end{tabular}

	Both keys and values are metavalues, and can thus be assigned any supported type (booleans,
	integers, reals, strings, enums, objects).
	
	If the tag is bound to a member of the target class, its value will be dynamic and
	thus depend on every instance. The valid bindings for dynamic tags are detailed in section
	\ref{sec:metaproperty-bind}.

	Metaproperties and metafunctions can hold any number of tags. They must provide the following interface to
	give access to them:
	
	\begin{itemize}
		\item Getting the total number of tags
		\item Retrieving a tag by its index
		\item Checking existence of a tag by its identifier
		\item Getting a tag's value by its identifier
	\end{itemize}

\section{Error handling}
\label{sec:error}

	Any inconsistent operation must be reported by CAMP as an error. A CAMP error must provide
	all the information about the error, so that the client code can interpret and format it
	in its own way.

	Here is the list of errors triggered by CAMP:

	\begin{description}
		\item[Invalid class] The requested metaclass does not exist.
		\item[Invalid enum] The requested metaenum does not exist.
		\item[Invalid value] The metavalue cannot be converted to the requested type.
		\item[Invalid argument] The metavalue cannot be converted to the requested metafunction parameter.
		\item[Invalid property] The requested metaproperty does not exist in the metaclass.
		\item[Invalid function] The requested metafunction does not exist in the metaclass.
		\item[Invalid enum value] The requested value does not exist in the metaenum.
		\item[Invalid enum name] The requested name does not exist in the metaenum.
		\item[Invalid object] The metaobject cannot be converted to the requested type.
		\item[Invalid index] The given index is out of the valid range.
		\item[Invalid access] The requested action cannot be executed due to access restrictions (metaproperty get/set, metafunction call).
	\end{description}

\section{Observers}
\label{sec:observer}

	CAMP must provide a way for the client code to be notified of some internal events.
	
	In particular, the system must notify the following events:
	
	\begin{itemize}
		\item Metaclass creation
		\item Metaenum creation
		\item Metaclass removal
		\item Metaenum removal
	\end{itemize}

\section{Visitors}
\label{sec:visitor}

	Although the basic API of CAMP provides everything necessary to build components on top of it, it
	also provides a framework to make this process more straight-forward.
	
	This framework is based on the visitor pattern, and makes exploring a metaclass' properties and
	functions easier and cleaner. In particular, it must give access to the actual type of properties
	(see \ref{sec:metaproperty-derived}), which is not possible with the standard metaclass and metaproperty
	interface.

	Here is an example of what a metaclass visitor must look like:
	
\begin{cppcode}
class ClassVisitor
{
public:
  void visit(Class);
    
protected:
  void visit(SimpleProperty);
  void visit(ArrayProperty);
  void visit(EnumProperty);
  void visit(ObjectProperty);
  void visit(Function);
};
\end{cppcode}

\appendix

\chapter{Why not just using Qt's properties system?}

	Qt provides a similar system that could be used directly, but it fails to meet the following requirements:
	
	\begin{itemize}
		\item The system is very intrusive, and tied to Qt framework and tools (MOC)
		\item Properties cannot be bound to direct members nor member of member, only to a set of accessors
		\item Objects embedding properties cannot be valid types of properties for other objects
		\item Valid types of properties cannot hold properties themselves, and thus cannot be introspected
		\item Visibility of properties cannot change according to the state of the object
		\item Properties cannot hold custom tags
		\item Multiple inheritance is not supported
	\end{itemize}

\chapter{Future improvements}

\section{Adding scopes for metaclasses and metaenums}

	This would mainly be used in script engines, in order to add a high-level layer
	of organization. Metaclasses and Metaenums could be declared in a given scope,
	which is either a metaclass or a ``namespace''.

\section{Support for dictionary properties}

	If necessary, CAMP could support a new type of properties bound to
	dictionaries (std::map, ...). It would define a specific interface to
	manipulate keys and values, as well as an extension system similar to
	arrays for binding custom types of dictionaries.

\section{Extending the Visitor base class}

	The Visitor class provides a minimal interface for exploring properties and functions
	of a metaclass. If required, it could be extended to include more advanced features such
	as filtering or custom traversal.

\section{Automatic support for \texttt{toString()} function}

	At declaration time, a metaclass could try to detect an available conversion from its target class
	to a string type, and automatically provide it in its interface through a \verb@toString(object)@ function.
	
	CAMP could detect the following typical implementations of the conversion:
	\begin{itemize}
		\item A operator$<<$(std::ostream\&, const T\&) function
		\item A T::toString() function
		\item A T::operator const char*() function
		\item A T::operator std::string() function
	\end{itemize}
	
	If a conversion exists but uses something else, it could also be given explicitely when declaring the metaclass.

\listoffigures

\end{document}
